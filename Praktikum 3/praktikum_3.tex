%Präambel
\documentclass[paper=a4,fontsize=11pt,headsepline,footsepline,parskip=half]{scrartcl}
\usepackage[utf8]{inputenc}
\usepackage[T1]{fontenc}
\usepackage{amsmath,amsfonts,amssymb}
\usepackage{ngerman,graphicx,textcomp,mathpazo,booktabs}
\usepackage[decimalsymbol=comma,per=frac]{siunitx}
\usepackage[textfont=sl,labelfont=bf]{caption}

%Seite einrichten
\areaset[2cm]		% Zusätzlicher Rand für die Bindung
        {17cm}{24cm}	% Textbreite und -Höhe

%Zeilenabstand
\linespread{1.2} %Standardwert

%Kopf- und Fußzeile
\usepackage{scrlayer-scrpage}
\setlength{\headheight}{23pt}
\lohead{\includegraphics[width=0.7cm]{../logofbi} Hochschule Darmstadt}
\rohead{David Falk, Christian Lichtsinn}
\pagestyle{scrheadings}

%Programmierzeilen
\usepackage{listings}
%Optionen für listings
\lstset{
frame=single, %Rahmen
numbers=left %Zeilennummer 
}

%sollte als letztes Paket geladen werden
\usepackage{hyperref}

\begin{document}

%Titelblatt
\begin{titlepage}

\begin{minipage}[c]{5cm}
\includegraphics[width=5cm]{../logofbi}
\end{minipage}
\hfill
\begin{minipage}[c]{10cm}
\begin{flushright}
\Large Einführung in die Technik\\und Anwendung von\\
\LARGE \textbf{RFID}
\end{flushright}
\end{minipage}

\vspace*{1cm}

\begin{minipage}[c]{9cm}
\begin{flushleft}
\large David Falk (736532)\\Christian Lichtsinn (736787)\\Praktikum 3: 16.11.15: \textbf{Mo-56x}
\end{flushleft}
\end{minipage}
\hfill
\begin{minipage}[c]{7cm}
\begin{flushright}
\large Betreuer:\\Prof. Ralf S. Mayer\\F. Dotzauer
\end{flushright}
\end{minipage}

\vspace*{1cm}

%Workaround nötig wegen parskip Option.
\begingroup
  \setlength{\parskip}{0pt}% keinen Absatzabstand einfügen
  \setlength{\parindent}{0pt}% nicht einziehen
  \setlength{\parfillskip}{0pt plus 1fil}% Absatz darf komplett gefüllt sein
  \par\rule{\linewidth}{1.5pt}\par
\endgroup

\vspace*{1cm}

\noindent
\large{Thema: \textbf{RFID-LF-Reader und Transponder}}

%KAPITEL 1
\section{Fragen zu Atmel ATA2270-EK1}

\subsection{Wann darf das Reader-Board erst eingeschaltet werden?}

Erst wenn Reader-Board und die $125 kHz$-Antenne korrekt am Mainboard angebracht wurden, darf die Stromversorgung eingesteckt werden (und damit das Board eingeschaltet).

\subsection{Welche Spannung an der Spule ist zu erwarten?}

Es ist eine Spannung von $200 V$ zu erwarten.

\subsection{Wie ist die Induktivität der Antennenspule? Wie groß etwa wäre die dazugehörige Kapazität auf der Leser-Platine?}

Die Induktivität der Antennenspule wird mit $700 \mu H$ angegeben. Bei einer Resonanzfrequenz von $\omega_r = 125 kHz$ ergibt sich mit

\end{titlepage}

\begin{align}
 \omega_r = \frac{1}{\sqrt{LC}} \Leftrightarrow C = \frac{1}{L \cdot \omega_r^2}
\end{align}

für die Kapazität $C = 91,429 nF$ auf der Leser-Platine.

\subsection{Was wäre zu beobachten, wenn eine andere Antennenspule mit unterschiedlicher Induktivität angeschlossen würde?}

Man müsste einen anderen Kondensator wählen, damit die Resonanzfrequenz $\omega_r = 125 kHz$ erreicht wird.

\subsection{Ist die mitgelieferte Antenne optimal?}

Die mitgelieferte Antenne ist eine Beispielantenne und nicht zB. für Reichweite optimiert.

\subsection{Woraus besteht ein Tag (Transponder) im vorliegenden Kit?}

Ein Tag besteht aus einem integrierten Schaltkreis (intregrated circuit, IC), einem Kondensator und einer Antennenspule.

\subsection{Wie ist die Reihenfolge der gespeicherten und übertragenen Bits? Welcher Endian? Little oder Big?}

Die Blockreihenfolge ist von links nach rechts von 1 bis 32. Big Endian.

\subsection{Muss ein Tag initialisiert werden? Wann?}

TODO (Ja? Vor dem Auslesen? Konfigurieren?)

\subsection{Welche Eigenschaften hat ein TK5551-Transponder?}

Ein TK5551-Transponder hat einen Konfigurationsblock und 7 Datenblöcke, jeder Datenblock speichert 32 bits und damit insgesamt 224 bits.

\subsection{Welche Eigenschaften hat ein ATA5577-Transponder?}

Ein ATA5577-Transponder hat einen Konfigurationsblock, 7 Datenblöcke à 32 bits (zusammen 224 bits) und 2 ID-Blöcken, sowie ein konfigurierbares
Analog-Frontend.

\subsection{Welche Besonderheiten hat ein ATA5570-Transponder?}

Je nachdem, ob die Impedanz hoch oder niedrig ist (was man mit dem Jumper J2 einstellen kann), werden die Daten korrekt oder invers versendet.

%KAPITEL 2
\section{Fragen zu SamSys-UHF-Reader}

\subsection{Wie ist das Lesegerät MP9320 in Betrieb zu nehmen, was ist zu beachten, wie wird die Antenne angeschlossen?}

Das Lesegerät verbindet man mit einem seriellem Kabel mit einem PC, damit man via einer Software mit dem Lesegerät kommunizieren kann. Das MP9320
darf dabei aber erst in Betrieb genommen werden, wenn entweder an den Antennenanschlüssen entsprechende UHF-RFID-Antennen angeschlossen sind oder
sich ein $50\ohm$-Abschlusswiderstand anstatt der Antenne befindet.

\subsection{Könnten mehrere UHF-Lesegeräte sich gegenseitig beeinflussen?}

TODO (Ja? Nein?)

\subsection{Was bedeutet ISO 18000-6?}

ISO 18000-6 ist eine Norm für die Spezifikation der Luftschnittstelle im UHF-Bereich (Ultra High Frequency) von $860 MHz$ bis $960 MHz$. 

\subsection{Wie können Tags mit der RF Command Suite und der RS232-Schnittstelle gelesen werden?}

TODO

%KAPITEL 3
\section{ATA2270-EK RFID-Reader Praktikumsaufgaben}

select tag/reader
read/write menu
block beschreiben und einzeln bestätigen

\subsection{TK5551 und ATA5570 Transponder am Reader-Board}

Damit der Reader den Transponder auslesen kann, muss er erstmal entsprechend eingestellt werden.

Wir konnten den TK5551 Transponder auslesen, ihn neu beschreiben und den Transponder einer andernen Gruppe auslesen mit den Daten, die
diese in die Blöcke geschrieben haben.

Den ATA5570 Transponder konnten wir ebenfalls erfolgreich auslesen und wie erwartet wurden mit dem Abstecken von Jumper J2 die Daten
in den Blöcken invertiert.

\subsection{ATA2270-EK1 via RS232-Schnittstelle am PC}

Einstellen: local echo, senden und antwort: CR+LF (carriage return + line feed)

\subsection{Softwareversionsabfrage}

\begin{lstlisting}[caption={Softwareversionsabfrage}]
 CMREV
 OK 0019 SW v3.4 Dec 17 2009
\end{lstlisting}

\subsection{HF anschalten und Transponder-Typ einstellen}

CMRFC1111ON

Transponder-Typ hätte mitgelieferte

CMSRT 5551

eingestellt werden müssen, hat aber nicht funktioniert, daher am Reader selbst eingestellt.

\subsection{Tag-Test}

CMKTIF

--> OK 0007 No Tag

--> OK 0011 Tag Present

\subsection{Tag auslesen}

RDRTS 32

RDRSS 32

liest die ersten 32 bit aus.

\subsection{Modulationsart ändern}

RDSRM0000MAN 

--> 000EEEEE

RDSRM0000BP1

--> FFFEFFFF

RDSRM0000BP2

--> 33300000

\subsection{ATA5577 ATA5570 Transponder via RS232-Schnittstelle am PC}

ANRMD (Herstellerdaten auslesen)

\section{Animal-ID: ISO 11748/11785}

Country: 0114 (276d), National: 16D75895C5, CRC 672F (Häkchen)

an rca --> OK 0030 1 0 0276 098102187461 672F OK

\section{SamSys-UHF-Reader}

Aufgrund von Software-Problemen konnten wir diese Testreihe nicht durchführen.

\end{document}
