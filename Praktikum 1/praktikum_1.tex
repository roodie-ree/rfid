\documentclass[11pt,a4paper,parskip=half]{scrartcl}
\usepackage[utf8]{inputenc}
\usepackage[T1]{fontenc}
\usepackage{amsmath, amsfonts, amssymb}
\usepackage{ngerman, graphicx, microtype, mathpazo, tabularx, xcolor}
\usepackage[decimalsymbol=comma,per=frac]{siunitx}
\usepackage[scale=0.7]{geometry}
\linespread{1.2}

\author{David Falk (736532), Christian Lichtsinn (736787)}
\title{RFID Praktikum 1}

\begin{document}
\maketitle

\section{OCR}
\section{Barcodes}
\subsection{Welche Arten von Barcodes kann das Gerät erfassen?}
\begin{itemize}
  \item EAN 8
\end{itemize}
\subsection{Wie viele Barcodes können in einer bestimmten Zeit erfasst werden?}
\subsection{Wie ändert sich die Erkennung mit Abstand, Winkel, Abdeckung, Einfluss von Licht?}
\subsection{Welche Auswirkung hätte Verschmutzung, Beschädigung des Barcodes?}
\subsection{Hängt die Lesegeschwindigkeit mit der Informationsmenge im Code zusammen?}

\section{Oszilloskop, Spektrumanalysator und Frequenzgenerator}

\section{Schwingkreis, Resonanz, Lastmodulation und -Demodulation}

\section{1,3GHz Mikrowellen}

\section{RFID HF}

\section{NFC}
\subsection{Was ist NFC?}
NFC ist eine Art RFID, bei der die stikte Trennung zwischen Lesegerät und Tag/Transponder aufgehoben wird.
NFC verwendet den HF-Bereich bei 13,56 MHz.
\subsection{Vorbereitete Tags}
\subsection{Tag konfigurieren}
\subsection{Übertragung zwischen zwei Smartphones}
Mit Android Beam ist es möglich Inhalte zwischen zwei Android-Telefonen übertragen.
Wir konnten eine URL übertragen, wenn eine Website geöffnet ist und den App Store
öffnen, wenn eine App geöffnet ist.
\end{document}
