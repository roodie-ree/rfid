%Präambel
\documentclass[a4paper,11pt,headsepline,footsepline,parskip=half]{scrartcl}
\usepackage[utf8]{inputenc}
\usepackage[T1]{fontenc}
\usepackage{ngerman,graphicx,mathpazo}

%Seite einrichten
\areaset[2cm]		% Zusätzlicher Rand für die Bindung
        {17cm}{24cm}	% Textbreite und -Höhe

%Zeilenabstand
\linespread{1.24}

%Linienstärke
\newcommand\HRule{\noindent\rule{\linewidth}{1.5pt}}

%Kopf- und Fußzeile
\usepackage{scrlayer-scrpage}
\setlength{\headheight}{23pt}
\lohead{\includegraphics[width=0.7cm]{logohdafbi-standalone} Hochschule Darmstadt}
\rohead{David Falk, Christian Lichtsinn}
\pagestyle{scrheadings}

%sollte als letztes Paket geladen werden
\usepackage{hyperref}

\begin{document}

%Titelblatt
\begin{titlepage}

\begin{minipage}[c]{5cm}
\includegraphics[width=5cm]{logohdafbi-standalone}
\end{minipage}
\hfil
\begin{minipage}[c]{10cm}
\begin{flushright}
\Large Einführung in die Technik\\und Anwendung von\\
\LARGE \textbf{RFID}
\end{flushright}
\end{minipage}

\vspace*{1cm}

\begin{minipage}[c]{8cm}
\begin{flushleft}
\large David Falk (736532)\\Christian Lichtsinn (736787)\\Praktikum 1 \& 2: 19.10.15 \& 02.11.15
\end{flushleft}
\end{minipage}
\hfil
\begin{minipage}[c]{8cm}
\begin{flushright}
\large Betreuer:\\Prof. Ralf S. Mayer\\F. Dotzauer, E. Wagner
\end{flushright}
\end{minipage}

\vspace*{1cm}

\HRule

\vspace*{1cm}

\noindent
\large Thema: \textbf{AutoID -- Resonanz -- Frequenzen -- RFID-Grundlagen}

\normalsize

\section{OCR}

Optical character recognition (OCR), bzw. Texterkennung ist eine Methode, mit der man mittels optischer
Eingabegeräte wie Scanner, Kamera oder auch Fax-Gerät Text auf Bildern erkennen kann und diesen dann auch
digital besser verarbeiten kann, sofern es denn einwandfrei funktioniert. Dass dem nicht immer so ist zeigt
unter anderem ein Beispielbild auf Wikipedia\footnote{\url{https://de.wikipedia.org/wiki/Texterkennung\#/media/File:Beispiel_Texterkennung.png}}
und ein Vortrag vom 31. Chaos Communication Congress, bei dem Informatiker David Kriesel über die Texterkennung
von Xerox-Scannern berichtet.\footnote{Traue keinem Scan, den du nicht selbst gefälscht hast [31c3] von David Kriesel:\\ \url{https://www.youtube.com/watch?v=Vp03vyNspyI}}

(Leider konnten wir aufgrund von fehlender Hardware keine Versuche zu Hause durchführen.)

\end{titlepage}

\section{Barcodes}

Barcodes sind parallel nebeneinander aufgestellte Striche verschiedener Dicke, die einen binären Code darstellen,
welcher mit Barcodelesegeräten ausgelesen werden kann. Angewendet werden diese Barcodes vor allem im Handel und Versand.

\subsection{Welche Arten von Barcodes kann das Gerät erfassen?}

Wir waren in der Lage, folgende Barcodes zu scannen: EAN-8, EAN-13, ISBN (was als EAN-13 ausgegeben wurde),
UPC-A, UPC-E, 3 of 9, Code39, Code128, Code EAN128 sowie Barcodes von Flaschen und den PCs im Labor, welche
sich als EAN-8/13 und Code39 ausgaben.

Zusätzlich gab es noch weitere Barcodes, die kurze Texte codiert hatten, die da waren: esaip Angers, h\_da, ?, fbi, Haardtweg, D-64295,
wwwfbi, IRT2, IRT3, Ralf Mayer, Email, enjoy the rfid lesson -und- good luck for all your scanning. (\textbf{TODO}: nochmal Text prüfen)

\subsection{Wie viele Barcodes können in einer bestimmten Zeit erfasst werden?}

Die Erfassung durch das Gerät geschieht unverzüglich. Da man das Gerät aber mit der Hand bedienen muss und den
Scanbereich zum Barcode führen muss, kann man ca. 1 Barcode pro Sekunde erfassen.

\subsection{Wie ändert sich die Erkennung mit Abstand, Winkel, Abdeckung, Einfluss von Licht?}

Der Barcode ließ sich geschätzt bis zu einem $^3/_4$ Meter Abstand vom Lesegerät erfassen. Bis zu einem geschätzten
Winkel von $80^\circ$ war der Barcode noch lesbar, solange alle Striche erfasst wurden. War der Barcode abgedeckt, war
für das Lesegerät keine Erkennung mehr möglich. Den Einfluss von Licht konnten wir nicht testen, da wir keine
sehr helle Lichtquelle zur Verfügung hatten, auch nicht das Sonnenlicht, da es am Versuchstag bewölkt war.

\subsection{Welche Auswirkung hätte Verschmutzung, Beschädigung des Barcodes?}

Da das Scannen von Barcodes eine optische Methode ist, ist jede optische Änderung des Barcodes, eben Verschmutzung oder Beschädigung des
Codes, ein Problem, die zu Schwierigkeiten oder aber zum Nicht-Erfassen des Barcodes durch das Lesegerät führt.

\subsection{Hängt die Lesegeschwindigkeit mit der Informationsmenge im Code zusammen?}

Die Erfassung des Codes erfolgt komplett und unverzüglich. Man kann auch nicht so viel Information in den Barcodes
speichern und es gab auch keinen erkennbaren Unterschied bei den Scans. Daher nein.

\section{Oszilloskop, Spektrumanalysator und Frequenzgenerator}

TODO

\section{Schwingkreis, Resonanz, Lastmodulation und Demodulation}

TODO

\section{1,3 GHz Mikrowellen}

\subsection{Lineare Polarisation}

Im Gegensatz zur zirkulären Polarisation schwingen die Wellen(?) nur in einer Ebene, zum Beispiel nur senkrecht oder nur waagrecht.
Das hat zur Folge, dass wenn das Empfangsgerät(?) nicht genauso wie der Sender ausgerichtet ist, dann wird das Signal mit jedem Grad Änderung
zur Ausrichtung des Senders schwächer, bis es nicht mehr zu empfangen ist bei $90^\circ$ Drehung.

\subsection{Absorption}

Je höher die Frequenz ist, desto mehr wird in Flüssigkeiten wie Wasser absorbiert. Das ist für LF noch kein Problem, für Mikrowellen aber schon.
Deswegen sind es vor allem \glqq Behälter\grqq mit Flüssigkeiten, wie Flaschen oder Menschen, die das Signal erheblich stören. Aber auch
Objekte, die ungefähr so groß wie die Sender-/Empfänger-Antennen sind oder das in Bleistiften enthaltene Graphit haben Einfluss auf das
empfangene Signal.

(\textbf{TODO} Welche Auswirkung auf RFID?)

\subsection{Signalstärke zu Abstand}

TODO

\section{RFID HF}

\section{NFC}

\subsection{Was ist NFC?}

NFC ist RFID, bei der die stikte Trennung zwischen Lesegerät und Tag/Transponder aufgehoben wird.
NFC verwendet den HF-Bereich bei 13,56 MHz.

\subsection{Vorbereitete Tags}

\subsection{Tag konfigurieren}

\subsection{Übertragung zwischen zwei Smartphones}

Mit Android Beam ist es möglich Inhalte zwischen zwei Android-Telefonen übertragen.
Wir konnten eine URL übertragen, wenn eine Website geöffnet ist und den App Store
öffnen, wenn eine App geöffnet ist.

\end{document}
