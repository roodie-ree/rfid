%Präambel
\documentclass[paper=a4,fontsize=11pt,headsepline,footsepline,parskip=half]{scrartcl}
\usepackage[utf8]{inputenc}
\usepackage[T1]{fontenc}
\usepackage{amsmath,amsfonts,amssymb}
\usepackage{ngerman,graphicx,textcomp,mathpazo,booktabs}
\usepackage[decimalsymbol=comma,per=frac]{siunitx}
\usepackage[textfont=sl,labelfont=bf]{caption}

%Seite einrichten
\areaset[2cm]		% Zusätzlicher Rand für die Bindung
        {17cm}{24cm}	% Textbreite und -Höhe

%Zeilenabstand
\linespread{1.2} %Standardwert

%Kopf- und Fußzeile
\usepackage{scrlayer-scrpage}
\setlength{\headheight}{23pt}
\lohead{\includegraphics[width=0.7cm]{../logofbi} Hochschule Darmstadt}
\rohead{David Falk, Christian Lichtsinn}
\pagestyle{scrheadings}

%Programmierzeilen
\usepackage{listings}
%Optionen für listings
\lstset{
frame=single, %Rahmen
numbers=left %Zeilennummer
}

%sollte als letztes Paket geladen werden
\usepackage[hidelinks]{hyperref}

\begin{document}

%Titelblatt
\begin{titlepage}

\begin{minipage}[c]{5cm}
\includegraphics[width=5cm]{../logofbi}
\end{minipage}
\hfill
\begin{minipage}[c]{10cm}
\begin{flushright}
\Large Einführung in die Technik\\und Anwendung von\\
\LARGE \textbf{RFID}
\end{flushright}
\end{minipage}

\vspace*{1cm}

\begin{minipage}[c]{9cm}
\begin{flushleft}
\large David Falk (736532)\\Christian Lichtsinn (736787)\\Praktikum 5 \& 6: 14.12.15 \& 18.01.16: \textbf{Mo-56x}
\end{flushleft}
\end{minipage}
\hfill
\begin{minipage}[c]{7cm}
\begin{flushright}
\large Betreuer:\\Prof. Ralf S. Mayer\\F. Dotzauer
\end{flushright}
\end{minipage}

\vspace*{1cm}

%Workaround nötig wegen parskip Option.
\begingroup
  \setlength{\parskip}{0pt}% keinen Absatzabstand einfügen
  \setlength{\parindent}{0pt}% nicht einziehen
  \setlength{\parfillskip}{0pt plus 1fil}% Absatz darf komplett gefüllt sein
  \par\rule{\linewidth}{1.5pt}\par
\endgroup

\vspace*{\stretch{1}} %stretch zählt all stretch zusammen (hier 1+2=3) und verteilt den vspace entsprechend, hier 1/3

\centering
\Huge{\textbf{\textsl{\huge Dokumentation zur RFID-HF-Applikation\\ \Huge Check das Gepäck!}}}

\vspace*{\stretch{2}} %und hier 2/3 vspace vom Rest der Seite.

\end{titlepage}

\tableofcontents

%KAPITEL 1
\section{Einleitung}

bla

\section{Installation}

bla

\section{Anwendung}

bla

\section{Code-Dokumentation}

bla

\end{document}
